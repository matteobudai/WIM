\section{Content}

Let's now move to the analysis of the content.
The page that we analyze is \url{https://siciliangoodness.shop/padovaviaroma/}. 
In general we can see that the back button works very well. From each page where we are we can go back to the previous page.
An other important aspect is that we avoid the phenomenon of 'The lost in the navigation'. In fact, in the homepage, with the breadcrumbs, we always know where we are and the path that we follow. 

\begin{figure}[H]
	\centering\includegraphics[width=12cm]{Img/bred2.png}
	\caption{Homepage Breadcrumb}
\end{figure}

The only pages that have some problems are 'SHOP VIGONZA' and SHOP PADOVA VIA ROMA'. Indeed, in these pages it is difficult to come back to the homepage, because it seems to be a new page where you start your research. We can see that no breadcrumbs are present and if we click on the logo we reload the same page. This is confirmed by the fact that if we click on a link (for example: 'VINI \& LIQUORI) we start the navigation from this page.

\begin{figure}[H]
	\centering\includegraphics[width=12cm]{Img/bred1.png}
	\caption{Shop Breadcrumb}
\end{figure} 

In particular in this section we'll analyze the page to search the products and the 404 page will be briefly analyzed too. In conclusion we'll analyze the page of a single product (with the analysis of the Ws), as it is the most significant and most visited type of page.

\subsection{Shop}
In this page the logo is well situated in the top-left corner of the website.
We don't deeply analyze the Ws but we can see in the image that the mandatory axis are present(Who: in the center, Where: in the center, What: in the left column).

\begin{figure}[H]
	\centering\includegraphics[width=12cm]{Img/Shop.png}
	\caption{Shop page}
\end{figure}

There is a lot of content in this page. In fact, for seeing all the information of the page we have to scroll more or less 7 times.
\newline
The menu is well structured vertically. In fact, the user can clearly search what he wants. 
Even the underlying menu is well structured. Indeed, whit its design it avoids unwanted clicks.

\begin{figure}[H]
	\centering\includegraphics[height=12cm, width=7cm]{Img/menushop1.png}
	\caption{Menu Shop page 1}
\end{figure}

\begin{figure}[H]
	\centering\includegraphics[height=12cm, width=12 cm]{Img/menushop2.png}
	\caption{Menu Shop page 2}
\end{figure}

To reload the page we can click on the logo or in the apposite icon under the logo.
On the right(of the previous icon) we find the registration button/icon. 
This is an other important factor because the registration is not mandatory and is not forced in any way.

\pagebreak

\subsection{Search}
Below these two icons we find the search functionality. 

\begin{figure}[H]
	\centering\includegraphics[height=12cm, width=10 cm]{Img/search.png}
	\caption{Search functionality}
\end{figure}

It is well structured and located in a good position.
If we use the search functionality, then we can obtain:
\begin{itemize}
	\item the results(Item);
	\item no results.
\end{itemize}

\pagebreak
  
\subsubsection{Item}
In this page we can see that the items are well locate.
Being an e-commerce the price and the products are very important.
We can observe that the website to attract users uses appropiate colors and good photos.

\begin{figure}[H]
	\centering\includegraphics[width=12cm]{Img/product.png}
	\caption{Products}
\end{figure}

As for the price, we immediately see the percentage of discount but not the price.
The only way to see the price is the hover event and this is not too good. It is better that the price is always visible because it is one of the most important things for a user.
However, The website puts the price close to the description and that is good. In addition, an other good thing is that you can see both the price and the discounted price.

\begin{figure}[H]
	\centering\includegraphics[width=12cm]{Img/price.png}
	\caption{Price}
\end{figure}


\subsubsection{No results}
If a page is searched through the search functionality that is not present within the site, it is displayed a clear error message.
For example we looked for 'milk' but this result doesn't exist. The website provides a good solution even if it not provides a description.

\begin{figure}[H]
	\centering\includegraphics[width=12cm]{Img/noresult.png}
	\caption{No results page}
\end{figure}

\subsection{404}
When the average user ends up on a non-existent page, it needs simple explanations, without the use of technicalities.
If a page is searched through the URL that is not present within the site, it is displayed a clear error message.
For example we looked for \url{https://siciliangoodness.shop/padovaviaroma/vio} but this page doesn't exist. The website provides a good solution telling that the page doesn't exist.

\begin{figure}[H]
	\centering\includegraphics[width=12cm]{Img/404.png}
	\caption{404 page}
\end{figure}

\pagebreak

\subsection{Product}
The page that we analyze is \url{https://siciliangoodness.shop/padovaviaroma/rosticceria/2-arancino-al-pistacchio.html}.

\begin{figure}[H]
	\centering\includegraphics[width=12cm]{Img/internal.png}
	\caption{Internal page}
\end{figure}

Usually, the internal pages do not necessarily have to answer all 6 Ws:
\begin{itemize}
	\item Where: Mandatory;
	\item Who: Mandatory;
	\item What: Mandatory;
	\item When: Optional;
	\item Why: Optional;
	\item How: Optional.
\end{itemize}
As for the homepage to analyze the internal page we analyze the so-called "Six Ws".

\subsubsection{The Six Ws}

\paragraph{Where}

\paragraph{Who}

\paragraph{What}

\paragraph{When}

\paragraph{Why}

\paragraph{How}

\pagebreak